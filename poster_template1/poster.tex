%%%%%%%%%%%%%%%%%%%%%%%%%%%%%%%%%%%%%%
% LaTeX poster template
% Created by Nathaniel Johnston
% August 2009
% http://www.nathanieljohnston.com/index.php/2009/08/latex-poster-template/
%%%%%%%%%%%%%%%%%%%%%%%%%%%%%%%%%%%%%%

\documentclass[10pt]{beamer}
\usepackage{times}
\usepackage{amsmath,amsthm,amssymb,latexsym}
\boldmath
\usepackage[english]{babel}
\usepackage[latin1]{inputenc}
\usepackage{color}
\usepackage{type1cm}
\usepackage[scale=1.15]{beamerposter}
\usepackage{graphicx}			% allows us to import images
\usepackage{caption}

% \usepackage{stfloats}
\usepackage{subfig}
% \usepackage{fixltx2e}

\definecolor{darkgreen}{rgb}{0.05,.3,0.05}

%-----------------------------------------------------------
% Define the column width and poster size
% To set effective sepwid, onecolwid and twocolwid values, first choose how many columns you want and how much separation you want between columns
% The separation I chose is 0.024 and I want 4 columns
% Then set onecolwid to be (1-(4+1)*0.024)/4 = 0.22
% Set twocolwid to be 2*onecolwid + sepwid = 0.464
%-----------------------------------------------------------

\newlength{\sepwid}
\newlength{\onecolwid}
\newlength{\twocolwid}
\newlength{\threecolwid}
\newlength{\fourcolwid}
%\setlength{\paperwidth}{72in}
%\setlength{\paperheight}{42in}
\setlength{\paperwidth}{36in}
\setlength{\paperheight}{48in}
\setlength{\sepwid}{0.024\paperwidth}
\setlength{\onecolwid}{0.301333\paperwidth}
\setlength{\twocolwid}{0.626666\paperwidth}
\setlength{\threecolwid}{0.952\paperwidth}
\setlength{\topmargin}{-0.5in}
\usetheme{confposter}

%-----------------------------------------------------------
% Define colours (see beamerthemeconfposter.sty to change these colour definitions)
%-----------------------------------------------------------

\setbeamercolor{block title}{fg=ngreen,bg=white}
\setbeamercolor{block body}{fg=black,bg=white}
\setbeamercolor{block alerted title}{fg=white,bg=dblue!70}
\setbeamercolor{block alerted body}{fg=black,bg=dblue!10}

%-----------------------------------------------------------
% Name and authors of poster/paper/research
%-----------------------------------------------------------

\title{}
\author{Jay High, Josh Osborne}
\institute{Computer Science Department, Western Washington University}

%-----------------------------------------------------------
% Start the poster itself
%-----------------------------------------------------------
% The \rmfamily command is used frequently throughout the poster to force a serif font to be used for the body text
% Serif font is better for small text, sans-serif font is better for headers (for readability reasons)
%-----------------------------------------------------------

%% BH: Figure 4.1 in Kolda and Bader is a good one to include

\begin{document}
\begin{frame}[t]
	\begin{columns}[t,totalwidth=\threecolwid] % the [t] option aligns the column's content at the top
		\begin{column}{\sepwid}\end{column}	% empty spacer column
		%% COLUMN ONE
		\begin{column}{\onecolwid}
			\begin{block}{Abstract}
				\large
				\rmfamily{
				\textbf{Motivation:} Current services that offer hyper parameter optimization are unwieldy or prohibitively expensive. 
				
				\vskip1ex
				\textbf{Goal:} Create a tool that is free, intuitive to use, and offers support for multiple optimization methods.
				} % end rmfamily
			\end{block}
			\vskip2ex
			\begin{block}{Background} 
			\large			
			\rmfamily{
				\begin{itemize}
					\item Hyperparameters are settings of a machine learning algorithm that cannot be trained
					\item Configurations of these hyperparameters can dictate the efficacy of a model
					\item PICTURE OF LEARNING RATE VS COST GOES HERE
				\end{itemize}
				
				%Insert first picture here
				\begin{itemize}
					\item Heuristic tuning is slow and requires expert level intuition 
				\end{itemize}
			}
			\end{block}
		\end{column}
		\begin{column}{\sepwid}\end{column}			% empty spacer column
		%%%% COLUMN TWO
		\begin{column}{\onecolwid}
			
			\vskip2ex
			\begin{block}{Grid Search}
			\rmfamily{
				Grid search splits the parameter space into a grid of points. These points are then exhaustively sampled. 	
				GRID SEARCH PICTURE GOES HERE
				% bulleted lists go here
			}
			\end{block}
			\begin{block}{Random Search}
			\rmfamily{
				Random search selects configurations at random from their selected ranges. It has been found to outperform heuristic tuning as well as Grid Search.
				RANDOM SEARCH PICTURE GOES HERE
				% picture here
			}
			\end{block}
			\begin{block}{Bayesian Optimization}
			\rmfamily{
				%In cases where training evaluations are extremely expensive, there is a need for faster tuning methods.
				Bayesian Optimization takes in all previous evaluations and creates a multivariate Gaussian Process to model the hyper parameter space. 
				It uses a mix of exploration and exploitation to generate future configurations with minimal use of expensive training evaluations. 
				GUASSIAN PICTURE GOES HERE
				%These solutions all generate a function that attempts to model the objective function. This model is then used to predict behaviour at future test points.
				%More points improve the model.
				%Solutions vary in the model and in how they trade off exploration vs. exploitation. 
				%Bayesian Optimization is a type of Surrogate Model solution which given a history of previous evaluations generates a gaussian process of all potential evaluations.
				%This process is then used to generate new future evaluations. Bayesian Optimization has been shown to beat out other popular methods of tuning. 
				
				% bulleted lists go here
			}
			\end{block}
		\end{column}
		\begin{column}{\sepwid}\end{column}			% empty spacer column
		%% COLUMN THREE
		\begin{column}{\onecolwid}
			\begin{block}{Our Product}
				\large
				\rmfamily{
				 Tinker is a web service that allows clients to utilize the aforementioned optimization algorithms without need for implementation. 
				 ARCHITECTURE GOES HERE
				 CLIENT API GOES HERE
				} % end rmfamily
			\end{block}
			\vskip2ex
			\begin{block}{Future Work}
			\rmfamily
			\large{
				We plan on implementing other state of the art optimization methods such as HORDE and Tree of Parzen's Estimators.
				
				% bulleted lists go here
			}
			\end{block}
			\begin{block}{References}
			\rmfamily
			\large{
				%PLEASE FIX :)				
				%Guassian Process [Digital image]. (n.d.). Retrieved May 10, 2017, from http://scikit-learn.org/0.17/modules/gaussian_process.html
				%Guassian Process [Digital image]. (n.d.). Retrieved May 10, 2017, from [Guassian Process]. (n.d.). Retrieved from http://blogs.sas.com/content/subconsciousmusings/2016/09/20/local-search-optimization-for-h yperparameter-tuning/

				% bulleted lists go here
			}
			\end{block}
		\end{column}
		\begin{column}{\sepwid}\end{column}			% empty spacer column
\end{columns}
\end{frame}
\end{document}